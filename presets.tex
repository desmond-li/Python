%fonts
\setCJKmainfont[BoldFont={DFMing Std W9},ItalicFont={DFMing Std W7}]{DFMing Std W5} %W5 body, W7 character, W9 content
\setCJKmonofont{DFKai-SB} %nominal mark
\setCJKsansfont[BoldFont={PMingLiU-ExtB}]{PMingLiU} %make up the missing character
\setmainfont{Times New Roman}
\setmonofont{Segoe UI Symbol} %font for symbols

%contents style
\renewcommand{\contentsname}{部\ \ 首\ \ 索\ \ 引}
\setcounter{tocdepth}{1} %only show chapter and section

%section style
\titleformat{\chapter}{\centering\LARGE\bfseries}{}{}{}
\titleformat{\section}{\centering\Large\bfseries}{}{}{}
\titleformat{\subsection}{\flushleft\LARGE\itshape}{}{}{}

%cancelling numbering the section
\makeatletter
\newcommand\specialsectioning{\setcounter{secnumdepth}{-2}}
\makeatother

%page style
\fancyhead[LO]{\nouppercase\leftmark}
\fancyhead[RO]{\nouppercase\rightmark}
\fancyhead[LE]{\nouppercase\rightmark}
\fancyhead[RE]{\nouppercase\leftmark}
\renewcommand{\headrulewidth}{0.4pt}
\fancyfoot[C]{\thepage}
\renewcommand{\footrulewidth}{0.4pt}
\pagestyle{fancy}

%set length
\setlength\columnseprule{0.4pt} %separation rule between columns
\setlength\parindent{0pt} %cancel indent

%entry format
\newcommand{\entry}[4]{\subsection{#1}\noindent\textbf{#2}\index{{\Huge{#3}}!{\large{#2}}!{\large{#1}}}\\ {#4}} % Defines the command to print each word on the page

%adjust item separation
\setenumerate[1]{itemsep=0pt,partopsep=0pt,parsep=\parskip,topsep=0pt}
\setitemize[1]{itemsep=0pt,partopsep=0pt,parsep=\parskip,topsep=0pt}

%circled text
\newcommand{\circled}[1]{\tikz[baseline=(char.base)]{\node[shape=circle,draw,inner sep=0pt] (char) {#1};}}

%index
\renewcommand{\indexname}{拼\ \ 音\ \ 索\ \ 引}
\makeindex